\documentclass{article}

% if you need to pass options to natbib, use, e.g.:
% \PassOptionsToPackage{numbers, compress}{natbib}
% before loading nips_2017
%
% to avoid loading the natbib package, add option nonatbib:
% \usepackage[nonatbib]{nips_2017}

\usepackage{nips_2017}

% to compile a camera-ready version, add the [final] option, e.g.:
% \usepackage[final]{nips_2017}

\usepackage[utf8]{inputenc} % allow utf-8 input
\usepackage[T1]{fontenc}    % use 8-bit T1 fonts
\usepackage{hyperref}       % hyperlinks
\usepackage{url}            % simple URL typesetting
\usepackage{booktabs}       % professional-quality tables
\usepackage{amsfonts}       % blackboard math symbols
\usepackage{nicefrac}       % compact symbols for 1/2, etc.
\usepackage{microtype}      % microtypography
\usepackage{amsmath}
\newcommand\RR{\mathbb{R}}
\def\norm#1{\Vert #1 \Vert}
\def\abs#1{\vert #1 \vert}
\title{CSE 190: Neural Networks 2017}

% The \author macro works with any number of authors. There are two
% commands used to separate the names and addresses of multiple
% authors: \And and \AND.
%
% Using \And between authors leaves it to LaTeX to determine where to
% break the lines. Using \AND forces a line break at that point. So,
% if LaTeX puts 3 of 4 authors names on the first line, and the last
% on the second line, try using \AND instead of \And before the third
% author name.

\author{
  %% examples of more authors
  %% \And
  %% Coauthor \\
  %% Affiliation \\
  %% Address \\
  %% \texttt{email} \\
  %% \AND
  %% Coauthor \\
  %% Affiliation \\
  %% Address \\
  %% \texttt{email} \\
  %% \And
  %% Coauthor \\
  %% Affiliation \\
  %% Address \\
  %% \texttt{email} \\
  %% \And
  %% Coauthor \\
  %% Affiliation \\
  %% Address \\
  %% \texttt{email} \\
}

\begin{document}
\maketitle

\begin{abstract}
    Machine learning programming practice on assignment 1.
\end{abstract}

\section*{Part I}
\subsection*{1.}
\subsubsection*{(a)}
For $d=2$, the activation rule is
\begin{align}
    && y &= \begin{cases} 1 & \text{if } w_{1}x_{1} + w_{2}x_{2} \geq \theta  \\ 0 &
        \text{ else} \end{cases} \notag
\end{align}
The equation for the line representing the decision boundary is therefore
\begin{align}
    && w_1 x_1 + w_2 x_2 &= \theta \notag 
\end{align}
or simply
\begin{align}
    && w^T x &= \theta \notag 
\end{align}
\subsubsection*{(b)}
Let $y$ be the projection of the origin onto the decision boundary. We must have
\begin{align}
    \tag*{($y$ must be on the boundary)}
    && y^T w &= \theta \notag \\
    \tag*{($y$ must be perpendicular to any vector parallel to the boundary)}
    && y^T w_{\perp} &= 0 & \forall w_{\perp}. w_{\perp}^{T} w = 0  \notag 
\end{align}
Thus, one can write,
\begin{align}
    && y &= \alpha w & \text{for some } \alpha \in \RR \notag 
\end{align}
So
\begin{align}
    && w^{T}y (w^{T} y)^{T} &= \theta \theta^{T} \notag \\
    \iff&& w^T y y^T w &= \theta^{2} \notag \\
    \iff&& w^T (\alpha w) (\alpha w^T) w &= \theta^{2} \notag \\
    \iff&& (\alpha^2 w^T w) (w^{T} w) &= \theta^{2} \notag \\
    \iff&& \norm{y}^2 \norm{w}^{2} &= \theta^{2} \notag \\
    \iff&& \norm{y}   &= \frac{\abs{w^T x}}{\norm{w}} \notag 
\end{align}
\end{document}
